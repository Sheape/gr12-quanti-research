\newsubsection{Conceptual Framework}

This study was mainly intended to determine the relationship of sleep quality
and blood sugar level in type 2 diabetes patients. Poor sleep quality may result
to poor blood sugar control, disrupt blood sugar levels, affect
appetite-regulating hormones, influence calorie intake, and impair
decision-making. This study also highlights the significant consequences of sleep
disruptions and provides valuable insights for healthcare institutions to better
treat their patients.

The Input-Process-Output (IPO) model delineates a system into three sequential
phases: input, processing, and output. In this model, inputs are depicted as
resources and actions introduced to a system during its initial cycle
stage, while outputs are portrayed as the outcomes generated by it
\parencite{maccuspie-2014}. To conduct a comprehensive examination to develop
effective strategies to enhance their inputs, the conceptual framework that will
be utilized is input-process-output (IPO) as seen in
\cref{fig:conceptual_framework}. This framework not only provides a structure
for conducting this study but also offers a visual representation of the
relevant variables and how they are interconnected in the study. It is crucial
to emphasize that the variables in this study are specifically associated with
patients diagnosed with type 2 diabetes, ensuring a precise and
focused analysis.

\begin{figure}[h]
  \begin{tikzpicture}[
    MODEL/.style={rounded corners, draw=black, node font=\small, text width=4.5cm, align=center, very thick, minimum size=3mm}, OUTPUT/.style={rounded corners, draw=black, node font=\small, text width=3.5cm, align=center, very thick, minimum size=3mm}]

    \node[MODEL] (input) {
     \setstretch{1}
     \textbf{Input}\\
     \begin{enumerate}
        \item Profile of Diabetes Patients which includes:
          \begin{enumerate}
            \item Age
            \item Gender
            \item Type of diabetes
          \end{enumerate}
        \item Sleep quality and Blood sugar level of respondents:
          \begin{enumerate}
            \item Sleep quality through PSQI
            \item Sleep disturbances experienced
            \item Sleep disorders (if any)
            \item Blood sugar level from medical record
          \end{enumerate}
     \end{enumerate}
   };
   \node[MODEL] (process) [right=of input] {
     \setstretch{1}
     \textbf{Process}\\
     \begin{enumerate}
        \item Collection of Related Literature
        \item Formulation of Questionnaire
        \item Requesting Permission from the School's Principal to Conduct the Study
        \item Distribution of Consent Forms
        \item Distribution of Survey
        \item Collection of Response from Survey Forms
        \item Statistical and Data Analysis
        \item Interpretation of Data
     \end{enumerate}
   };
   \node[OUTPUT] (output) [right=of process] {
     \textbf{Output}\\
     The profile of the respondents, their sleep quality, sleep disturbances,
     sleep disorders, blood sugar level and the relationship of the data gathered.
   };

    \draw[-{Latex[length=5mm,width=5mm]}] (input.east) -- (process.west);
    \draw[-{Latex[length=5mm,width=5mm]}] (process.east) -- (output.west);

  \end{tikzpicture}
  \caption{Conceptual Framework using Input-Process-Output}
  \label{fig:conceptual_framework}
\end{figure}


The input in the IPO model consists of the respondent's profile, the type of
diabetes they have, sleep quality in the past month, blood sugar level from
their medical record, sleep disorders (if any), and if they have other disease
or disorders.

The process consists of researchers collecting data from related literature and
generated questions based on \ac{psqi}, profile, and Likert scale to determine
the quality of sleep that the participants had. The independent variable for
this study is the sleep quality of type 1 and type 2 diabetes patients, while
the dependent variable is the blood sugar level. The data will be collected
digitally through Google Forms, and the sampling technique that will be used is
criterion sampling. This approach involves surveying individuals who meet
specific requirements, ensuring a focused and targeted data collection process.
The data will be gathered by creating a format \ac{psqi} questionnaire
consisting of 19-items, it is designed to measure sleep quality and disturbances
over the past few months as this research involves patients who experience sleep
disturbances \parencite{zhong-2015}.

The output in the model is the profile of the respondents, their sleep quality,
sleep disturbances, sleep disorders, blood sugar level and the relationship of
the data gathered. The researchers will determine the correlation of the
variables and data gathered by performing statistical analysis through the use
of R language.
