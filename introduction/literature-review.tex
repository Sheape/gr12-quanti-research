\subsection*{Literature Review}
\addcontentsline{toc}{subsection}{Literature Review}
Sleep disturbances have been found to have a significant impact on glycemic
control in individuals with type 2 diabetes (T2DM). The relationship between
sleep disturbance and glycemic control may vary depending on the presence of
complications of diabetes, such as painful neuropathy \parencite{zhu-2017}. In
patients with long-standing type 1 diabetes (T1DM), there is a decrease in sleep
quality and an increased prevalence of sleep disturbances. Reduction of sleep
duration and/or decreased sleep quality have been shown to impair glucose
tolerance and reduce insulin sensitivity in healthy individuals, and similar
negative effects on glucose metabolism may occur in patients with T1DM
\parencite{van-dijk-2011}.  There is a bidirectional relationship between sleep
disorders and T2DM. T2DM has been associated with a higher incidence of sleep
disorders, and shorter sleep duration and erratic sleep behavior have been
linked to a higher incidence of obesity, metabolic syndrome, and T2DM. Emerging
evidence suggests a relationship between sleep quality and glycemic control in
individuals with T2DM, highlighting the importance of assessing sleep quality
and sleep disorders as part of comprehensive medical evaluations
\parencite{khandelwal-2017}.  Sleep disturbances in individuals with T1DM have
been found to be associated with worse glycemic control and difficulties with
diabetes management \parencite{perfect-2020}. In children and adolescents with
T1DM, adequate and good quality sleep contribute to maintaining optimal glycemic
control, potentially through the impact of sleep on behaviors that drive
adherence to diabetes management regimens. Poor parental sleep quality has also
been associated with lower self-efficacy for diabetes management in children
with T1DM \parencite{perez-2018}.  Sleep apnea, a common sleep disorder, has
been found to be associated with glycemic control in T2DM. The severity of
obstructive sleep apnea (OSA) in rapid eye movement (REM) sleep has been shown
to be a stronger predictor of glycemic control than OSA severity in non-rapid
eye movement (NREM) sleep in adults with T2DM (Koren et al., 2014). The
hypoxemic burdens of OSA have also been reported to be associated with glycemic
control in T2DM \parencite{koren-2014}.  Short sleep duration and poor sleep
quality are common among individuals with T2DM, and several systematic reviews
have highlighted the impact of sleep quantity and quality on glycemic control in
both individuals with and without diabetes \parencite{smyth-2020}. In
individuals with T1DM, higher sleep variability has been associated with poorer
glycemic control. Hypoglycemia during sleep has also been found to significantly
increase sleep efficiency in children with T1DM \parencite{chontong-2016}.  In
conclusion, sleep disturbances have been found to have a significant impact on
glycemic control in individuals with both T1DM and T2DM. Adequate and good
quality sleep contribute to maintaining optimal glycemic control, while sleep
disturbances are associated with worse glycemic control and difficulties with
diabetes management. The severity of sleep disorders, such as obstructive sleep
apnea, can also affect glycemic control. Assessing sleep quality and sleep
disorders as part of comprehensive medical evaluations is recommended to
optimize glycemic control in individuals with diabetes.
