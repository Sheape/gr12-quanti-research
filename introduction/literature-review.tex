\subsection*{Literature Review}
\addcontentsline{toc}{subsection}{Literature Review}
The amount and quality of sleep a person has plays a huge role in Diabetes
\parencite{khandelwal-2017}. Over the past decade, the decrease in the amount of
sleep that people get has corresponded with a rise in the number of type 2
diabetes patients. With the increased number of occurrences of diabetes, it has
become alarming for both kids and adults. In addition to more common risk
factors for type 2 diabetes, such as being overweight, eating poorly, and not
exercising enough,  insufficient sleep, sleep of poor quality, and a disrupted
daily rhythm have been recognized as lifestyle factors that can contribute to
the development of type 2 diabetes \parencite{sondrup-2022}.

According to the research carried out by \textcite{van-cauter-2021}, diabetes
can affect the central nervous system, leading to changes in behavior,
neurotransmitter function, and autonomic processes. It can also disrupt hormone
functions, which can, in turn, lead to sleep problems. As a result, Sleep
disorders are significantly more common in people with diabetes as compared to
those without diabetes. Additionally, it states that irregular sleep patterns,
insufficient sleep, excessive sleep, and frequent awakenings during sleep
contribute to the development of glucose intolerance. As a result, the condition
of prediabetic and diabetic people will eventually worsen due to poor sleep.

A study conducted by \textcite{zhu-2017} stated that sleep disturbances have been
found to have a significant impact on glycemic control in individuals with type
2 diabetes mellitus (T2DM). The relationship between sleep disturbance and
glycemic control may differ based on whether individuals have diabetes-related
complications, like painful neuropathy. Furthermore, many people with T2DM
commonly experience both insufficient sleep duration and poor sleep quality, and
multiple systematic reviews have emphasized how the amount and quality of sleep
impact blood sugar control in individuals, regardless of whether they have
diabetes, as highlighted by \textcite{smyth-2020}.


Insufficient sleep duration and low sleep quality can potentially result in
reduced utilization of glucose by the brain, resulting in elevated blood sugar
levels. Sleep disruption may also contribute to changes in hormones that
regulate appetite, such as ghrelin and leptin. Furthermore, Sleep disturbance is
likely to enhance calorie intake, decrease energy expenditure, and lead to
compromised decision-making, such as choosing unhealthy foods and engaging in
sedentary behaviors. Consequently, this increases the risk of Type 2 diabetes or
poor glycemic control \textcite{zhu-2017}.

Subsequently, according to \textcite{amelia-2020}, the control of blood sugar
levels in people with type 2 diabetes mellitus is significantly correlated with
the quality of sleep. When compared to diabetic individuals who do not have
sleep problems, those who have sleep disorders are more likely to have elevated
blood sugar levels.

There is a bidirectional relationship between sleep disorders and T2DM. T2DM is
associated with an increased prevalence of sleep disorders, while shorter sleep
duration and irregular sleep patterns have been correlated with a greater
likelihood of developing obesity, metabolic syndrome, and T2DM. Recent findings
also indicate a potential link between the quality of sleep and how well
individuals with T2DM manage their blood sugar levels. This underscores the
importance of including assessments of sleep quality and the presence of sleep
disorders as integral components of thorough medical evaluations
\parencite{khandelwal-2017}.

Moreover, sleep apnea is a common sleep disorder that is associated with
glycemic control in T2DM. The severity of obstructive sleep apnea (OSA) in rapid
eye movement (REM) sleep is a stronger predictor of glycemic control than OSA
severity in non-rapid eye movement (NREM) sleep in adults with T2DM. The
hypoxemic burdens of OSA have also been reported to be associated with glycemic
control in T2DM \parencite{koren-2014}.

In addition, \textcite{shibabaw-2023} states that OSA syndromes have the
potential to cause harm to the liver and elevate the levels and function of
crucial liver enzymes, such as serum aspartate aminotransferase, alanine
aminotransferase, and alkaline phosphatase. Repeated episodes of OSA lead to
conditions like liver steatosis, necrosis, and inflammation, characterized by
the buildup of neutrophils and collagen deposits. This, in turn, disrupts
glucose regulation, even among individuals with Type 2 diabetes.

As stated by \textcite{pathak-2019}, The lack of physical activity and inactive
routines among people with diabetes contribute to obesity and impact their heart
and lung fitness. When individuals with diabetes change their eating habits and
use medications, it can significantly influence their sleep quality, often
negatively.

On top of that \textcite{perez-2018}, declared that, for children and adolescents
with T1DM, adequate and good-quality sleep contributes to maintaining optimal
glycemic control, potentially through the impact of sleep on behaviors that
drive adherence to diabetes management regimens. Furthermore, poor parental
sleep quality has also been associated with lower self-efficacy for diabetes
management in children with T1DM.

In accordance with \textcite{perfect-2020}, Sleep disturbances in individuals
with T1DM are associated with worse glycemic control and difficulties with
diabetes management. On top of that, in patients with long-standing type 1
diabetes (T1DM), there is a decrease in sleep quality and an increased
prevalence of sleep disturbances. Reduction of sleep duration and/or decreased
sleep quality have been shown to impair glucose tolerance and reduce insulin
sensitivity in healthy individuals, and similar negative effects on glucose
metabolism may occur in patients with T1DM \parencite{van-dijk-2011}.

In individuals with T1DM, higher sleep variability has been associated with
poorer glycemic control. Hypoglycemia during sleep has also been found to
significantly increase sleep efficiency in children with T1DM \parencite{chontong-2016}.

In summary, it is clear that sleep disturbances have been found to have a
substantial impact on glycemic control in individuals with both T1DM and T2DM.
Adequate and good quality sleep contributes to maintaining optimal glycemic
control, while sleep disturbances are associated with worse glycemic control and
difficulties with diabetes management. Furthermore, the severity of sleep
disorders, such as obstructive sleep apnea, can further complicate glycemic
control. Therefore, it is advisable to include assessments of sleep quality and
the presence of sleep disorders as part of comprehensive medical evaluations for
individuals with diabetes, with the aim of improving their ability to manage
their blood sugar levels \parencite{hamasaki-2016}.

Diabetes represents a pressing global health concern, with a pronounced upswing
anticipated in Asia, including the Philippines, by 2025. Noncommunicable
diseases (NCDs), including diabetes, now stand among the foremost causes of
mortality, driven partly by adopting Western lifestyles
\parencite{who-2023-noncommunicable}. These lifestyle-related diseases encompass
cardiovascular conditions, malignant neoplasms, diabetes, and chronic
respiratory illnesses \parencite{cockerham-2017}.

On the global stage, diabetes assumes a significant role as the 9th leading
cause of death and the 8th leading cause of disability, with mortality risk
escalating alongside factors such as age, disease complications, comorbidities,
and polypharmacy \parencite{who-2023-mortality}. The phenomenon of
multimorbidity, the coexistence of multiple diseases within a single individual,
prevails extensively among the diabetic population, especially within older
cohorts. This multifaceted burden is correlated with an elevated susceptibility
to cardiovascular complications and geriatric syndromes such as falls, cognitive
impairment, and urinary incontinence \parencite{kirkman-2012}. Projections suggest
a marked surge in diabetes prevalence, especially in low and middle-income
nations.

In a study conducted by \textcite{zaraspe-2017} concerning the relationship
between sleep quality and glucose control, it has been shown that sleep quality
is weakly associated with glucose control and that patients diagnosed with T2DM
who have poor glucose control are more likely to have poor sleep quality. 241
adult respondents with T2DM were measured using Pittsburgh Sleep Quality Index
(PSQI), and 55\% suffer from sleep disorders and 70.45\% suffer from low
glycemic control, resulting in poor sleep quality. For patients with T2DM, based
on PSQI scoring, poor sleep quality is worsened by HbA1c, obstructive airway
diseases, and sleeping alone in the bedroom.

According to \textcite{pelegrino-2021}, in the 2016 Healthy Living Index survey,
Filipinos suffer from sleep deprivation, one of the highest rates in Asia. 46\%
of Filipinos do not sleep enough, while 32\% said they sleep less than six
hours. On the other hand, according to a study by \textcite{singh-2020},
Filipinos now sleep 6 hours and 25 minutes a day. Sleeping habits such as sleep
duration are highly influenced by socio-economic factors in different countries
around the globe \parencite{lajunen-2023}. Both studies have different results
in different years, showing that sleeping practices in the Philippines vary from
time to time, resulting in significant changes and inconsistencies in the
results.

Within the Philippines, Type 2 diabetes (T2D) predominates, and the nation
confronts a burgeoning T2D prevalence. Notably, the 2013 Eighth National
Nutrition Survey reported a 5.4\% prevalence of high fasting blood glucose (>125
mg/dL) among adults aged >20 years, marking an increase from 2008. This surge is
particularly pronounced among affluent demographics, urban populations, and
individuals aged 60 to 69. Aligned with the global trajectory, the International
Diabetes Federation estimated a staggering 3.2 million T2D cases in the
Philippines in 2014, with a prevalence rate of 5.9\% among adults aged 20 to 79.
Importantly, a substantial number of cases remain undiagnosed. The economic
ramifications are tangible, with an estimated cost of
\$205 per person with T2D in 2013, a figure that mirrors those of neighboring
countries such as Thailand and Indonesia \parencite{tan-2016}.

Apart from this, in the study of \textcite{de-leon-2021}, the overall prevalence
of OSA-HR among patients with uncontrolled type 2 diabetes is 58.33\%, based on
240 respondents (88 males and 151 females). The majority of OSA-HR patients
(105/140) were above the age of 46. Patients with uncontrolled diabetes have a
significant prevalence of high OSA risk. HbA1c, dyslipidemia, BMI, Mallampati
score, tonsillar grade, and Epworth score are all related to a greater risk of
OSA in those with uncontrolled diabetes. This demonstrates that the prevalence
of HR-OSA rises with age. Regardless of whether or not they have OSA, elderly
patients are more likely to acquire sleep disturbances. Females also have a
greater prevalence rate of OSA-HR with 69.11\% compared to males, who have a
58\% likelihood.

On the basis of \textcite{agarwal-2019}, 50 of the 200 participants—25 from a
public diabetes clinic and 25 from each of three private diabetes clinics in
Zamboanga City—were diagnosed with diabetes. The recruitment process resulted in
150 volunteers without a diabetes diagnosis. The study was completed by all
respondents. In the undiagnosed population, the combined prevalence of T2DM and
prediabetes was 14\% (21/150). Nine (6\%) people had diabetes, whereas twelve
(8\%) had prediabetes. The groups' median ages were slightly different, with
those with recognized diabetes being a little older and those with normal blood
sugar being the youngest. Prior research done in the Philippines found that the
overall prevalence of dysglycemia was 14.2\% (7.2\% diabetes and 7.0\%
prediabetes), and that the prevalence rate of undiagnosed dysglycemia (diabetes
or prediabetes) in this group was 14\%.

Considering this, the Pittsburgh Sleep Quality Index (PSQI) is a quick and
simple tool that evaluates a person's subjective experience of his or her sleep
and supplements it with objective sleep measurement, broadening the scope of the
sleep evaluation. The PSQI remains the most commonly used tool to determine
subjective sleep disruption. It has been translated into various languages and
verified across a variety of populations. It evaluates seven distinct
sleep-related factors, such as daytime dysfunction, habitual sleep efficiency,
sleep disturbances, use of sleep medications, sleep quality, latency, and
duration. Both objective and subjective methods of measuring sleep are now
commonly employed due to technological advancements. Polysomnography (PSG) and
actigraphy are two such methods. However, their extensive training requirements,
expensive expenses, and inability to measure subjective sleep, limit their
utilization \parencite{zhu-2018}.

In consonance, disturbances such as nocturnal waking, nighttime toilet use, and
breathing problems may be detrimental to diabetic management in light of the
found connection between the PSQI sleep disturbances sub-scale and HbAlc in this
investigation. Such disruptions may be linked to a number of illnesses that
frequently coexist with diabetes, such as side effects of medications (such as
insulin-related hypoglycemia), polyuria linked to hyperglycemia, and Obstructive
snoring \parencite{telford-2018}.

In contrast, according to the study of \textcite{garber-2017}, glycemic
management is encouraged in addition to this to lessen microvascular
consequences. The development of T2D and its related macrovascular complications
are attributable to underlying risk factors for obesity and prediabetes,
according to this document. The algorithm also offers suggestions for
controlling cholesterol and blood pressure, two of the most significant risk
factors for cardiovascular disease (CVD). The gradual pancreatic beta-cell
dysfunction that causes metabolic regulation to progressively deteriorate over
time is now known to start early and may exist even before the diagnosis of
diabetes. In order to construct an exercise prescription for each patient based
on their objectives and restrictions, it is important to evaluate patients with
diabetes, extreme obesity, or other comorbidities for contraindications or
limitations to increased physical activity. Garber also stated that sleep
deprivation exacerbates insulin resistance, hypertension, hyperglycemia, and
dyslipidemia, and elevates inflammatory cytokines, whereas research suggests
that getting 6 to 9 hours of sleep every night can lower risk factors for heart
disease and diabetes. Furthermore, recent psychological intervention
meta-analyses shed light on effective strategies. The last step in lifestyle
therapy is quitting, which requires abstaining from all tobacco products. For
patients who are having trouble quitting smoking, nicotine replacement treatment
should be taken into consideration. For more stubborn patients who are unable to
quit smoking on their own, structured programs should be suggested.
