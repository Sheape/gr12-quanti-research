\newsubsection{Background of the Study}

Diabetes is a chronic disease that arises from a lack of insulin, a hormone
produced by the pancreas to control blood production, or from the body's
inability to properly utilize insulin. Diabetes considerably worsens or causes
serious heart attacks, strokes, kidney issues, and amputations of lower limbs.
The two types of chronic diabetes are type 1 and type 2
\parencite{who-2023-diabetes}. The \textcite{niddk-2023} defines type 1 diabetes
as a chronic disease in which the body's immune system, which normally fights
infections, targets and eliminates insulin-producing pancreas cells. This causes
the pancreas to stop producing insulin. Without insulin, glucose cannot enter
cells, which causes blood sugar levels to rise beyond the normal range. Type 2
diabetes is a condition that arises from an issue in the body's control and
utilization of glucose as an energy source. In type 2 diabetes, the body does
not produce sufficient insulin or does not use it effectively
\parencite{chundawat-2022}. This leads to an excess of sugar circulating in the
blood, and the cells do not receive an adequate amount. Over time, high blood
sugar levels can result in complications affecting the circulatory, nervous, and
immune systems. More than 95\% of people with diabetes have type 2 diabetes,
which used to be found predominantly in adults but is now becoming more
prevalent in children.

According to \textcite{khan-2019}, there were 6059 cases of type 2 diabetes
worldwide in 2017, affecting roughly 462 million people, or 6.28\% of the world's
population (4.4\% of people aged 15 to 49, 15\% of those aged 50 to 69, and 22\% of
people over 70 years of age). Diabetes is the 9th leading cause of death and the
8th leading cause of disability worldwide. The risk of dying from diabetes
increases with age, disease complications, comorbidities, and polypharmacy \parencite{who-2023-mortality}. The prevalence of
diabetes mellitus is increasing worldwide and in developed regions like Western
Europe, it is growing much more quickly. The incidence peaks at about 55 years
of age, and the gender distribution is equal. By 2030, the global prevalence of
type 2 diabetes is expected to increase to 7079 cases per 100,000 people,
showing an ongoing rise in all geographical areas. In lower-income nations,
there is an alarming risk of increased prevalence. Preventive clinical and
public health actions are required immediately. In Asian nations such as the
Philippines, the prevalence of diabetes is rising at an alarming rate. The
incidence and prevalence of type 2 diabetes (T2D) are both rising, and the
prevalence of prediabetes is likewise rising. There are 3.2 million cases of
type 2 diabetes (T2D) in the Philippines, with a 5.9\% prevalence rate in adults
between the ages of 20 and 79 years. It is important to note that around 1.7
million people with T2D remain undiagnosed \parencite{tan-2016}.

In 2014, 8.5\% of adults over the age of 18 were diagnosed with diabetes. In
2019, diabetes was the main cause of 1.5 million deaths, and 48\% of
diabetes-related deaths occurred before the age of 70. Another 460,000 kidney
deaths were caused by diabetes and high blood glucose, or about 20\% of
cardiovascular deaths. Between 2000 and 2019, the age-specific mortality rate of
diabetes increased by 3\%. In low- and middle-income countries, diabetes
mortality has increased by 13\%. Between 2000 and 2019, the probability of dying
of four major non-communicable diseases (cardiovascular diseases, cancer,
chronic respiratory diseases, and diabetes) among people aged 30 to 70 worldwide
declined by 22\% \parencite{who-2023-mortality}. According to the study
conducted by \textcite{paluyo-2022}, the prevalence of type 2 diabetes
among adults in the Philippines is predicted to be 7\%, with about 4 million
cases accounting for 6.5\% of total deaths in 2020.

Sleep disorders and disturbances are associated with people suffering from
diabetes. Nocturia, nocturnal hypoglycemia, peripheral neuropathy, restless leg
syndrome, and sleep-disordered breathing are all connected with sleep
disturbances in diabetes patients. When these disorders coexist with diabetes,
diabetic people can experience sleep disturbances and a poor quality of life
\parencite{surani-2015}. Type 2 diabetes mellitus (T2DM) is associated with a
higher incidence of sleep disorders, possibly caused by the disease itself or by
secondary complications or associated comorbidities associated with diabetes.
Sleep disorders are significantly more common in diabetic people than in
non-diabetic people. Several factors may contribute to insomnia in diabetics,
including peripheral neuropathy-related discomfort or pain, restless legs
syndrome, periodic limb movements, rapid changes in blood glucose levels during
the night leading to hypoglycemic and hyperglycemic episodes, nocturia, and
associated depression \parencite{khandelwal-2017}.

According to \textcite{morales-brown-2022}, evidence indicates a dual-sided
relationship between diabetes and sleep problems. In contrast, sleep
disturbances can have an impact on blood sugar levels and increase the chance of
developing insulin resistance. This is a reference to how blood glucose
regulation can make sleep worse. Furthermore, it is estimated that one in two
people with type 2 diabetes experience sleep issues as a result of their
unstable blood sugar levels and accompanying symptoms. Hyperglycemia (high blood
sugar) and hypoglycemia (low blood sugar) throughout the night can cause
insomnia and exhaustion the following day, making it difficult for diabetic
people to engage in physical activity. The kidneys overcompensate by making
diabetic people urinate more frequently when blood sugar levels are high. These
repeated potty visits during the night cause sleep disruptions. In addition to
these symptoms, high blood sugar can also result in thirst, headaches, and
fatigue, which makes it difficult to go to sleep. Conversely, regular exercise,
on the other hand, can help people with diabetes improve their sleep quality and
general health \parencite{pacheco-2023}.

Moreover, one of the things that affects glycemic control is physical activity.
The main mechanism of exercise-induced glycemic management is an increase in
whole-body insulin sensitivity. As determined by the person's medical history
assessment and physical examination, the long-term benefits of frequent exercise
on glycemic control seem to be related to the cumulative effect of temporary
gains in insulin sensitivity and glycemic control after each bout of exercise
rather than structural changes in insulin sensitivity \parencite{shiferaw-2022}.

Physical activities can significantly raise the risk of hypoglycemia in diabetic
patients who are using certain glucose-lowering medications, such as insulin and
sulfonylureas. Exercise-related hypoglycemia and fear of it are significant
concerns for diabetics. Exercises that have different effects on hypoglycemia,
such as aerobic and strength training, are frequently compared. The strength
training regimens can be thought of as an intense bout of physical activity,
whereas the aerobic exercise regimens recommended in the studies reported here
are of moderate intensity. Patients will need to assess their blood glucose pre-
and post-exercise to understand the demands, just like with all other areas of
diabetes management, in order to predict the reaction of the body \parencite{zahalka-2023}.

The principal objective of this study is to provide knowledge on the existing
knowledge base on diabetes by studying the correlation between sleep disorders
and blood sugar levels. Understanding the relationship between sleep disorders
and blood sugar levels can provide important indicators for improving blood
sugar control in diabetic patients. Researchers will be able to determine
whether sleep disturbance is the main cause of diabetes among non-diabetic
people, leading to further research into this topic.

This study aims to investigate the relationship between sleep quality and blood
sugar levels in both diabetic and non-diabetic patients in Angeles City,
Pampanga. It seeks to address specific questions regarding how sleep quality can
affect the risk of diabetes, the optimal sleeping habits to reduce blood sugar
levels, and how sleep disorders can intensify the effects of diabetes.
Ultimately, this research aims to contribute to a better understanding of the
role of sleep in diabetes, potentially providing insights into the prevention or
management of diabetes, especially in non-diabetic individuals. This study
investigates the interplay between sleep disturbances and blood sugar regulation
in individuals with diabetes within the context of Angeles City, Pampanga,
emphasizing the need for a comprehensive approach to diabetes care that includes
the management of sleep quality as a potentially influential factor.

Assessing the relationship between sleep disorders and disturbances and the
blood sugar levels of diabetic people can contribute globally in terms of aiding
diabetic people with their blood sugar level health and also giving caution to
non-diabetic people to have better sleep quality to prevent diabetes.
Understanding the sleep quality of diabetes patients can help lower their high
blood sugar levels. Estimating the impact of sleep quality on diabetic people
can help medical professionals give better management tips, thus nurturing the
field of medicine. Evaluating the sleep quality of diabetic patients can provide
insights and potentially enlighten the general public to prevent having
diabetes.
